% 1 Introduction:
\section{Introduction}
A panoramic image is a wide angle view of a particular scene. Panoramic imagery can be found in many places 
and has been used throughout history as a method to showcase large scenes in their entirety. Often, it is 
difficult to take photographs of a wide field-of-view without specialized equipement. As a result, research
in computer vision has identified methods in panorama stitching that combine multiple images with some 
overlap into a single panoramic image. Videos provide a source in which the likelihood of a panorama is high;
when a video pans, each frame overlaps with the previous to form a series of images with a wide field-of-view.
\par
Panorama stitching algorithms consist of two main steps; aligning images together by calculating homography
values between image pairs followed by blending them together to provide a clean transition between images.
Most algorithms also require that the input images have been identified as high quality and are supplied in 
the correct order ready for alignment and blending. Additionally, panoramas are restricted by the supply of
photographs of the physical destination. 
\par 
In these times, there exist vast caches of pictures and videos made available online through services such as
YouTube \cite{YouTube}, Vimeo \cite{Vimeo}, Picasa \cite{Picasa}, and many more. These resources can provide access
to multimedia content pertaining to many places around the world. Video sources have especially become common
for many of the popular places around the world, with many of these having the potential for becoming future
panoramas. Some of these videos are not of high quality and thus cannot be used to create a panorama. In an 
effort to identify these videos, we use three conditions; a video must cover a wide field-of-view, the video
must satisfy the thresholds for each image quality critereon, and that the video provides frames that relate
to each other through homographies and are not an individual collection of separate images.
\par
In this paper, we apply the strategy of panorama discovery in web videos identified in Liu et al \cite{Feng} 
with some improvements to shot detection and panorama discovery. A large part of our implementation focuses
on detecting panoramas compared with the majority of research being done on panorama stitching. This focus
is important as it allows users to automate the process of readying media for panorama stitching and provides
a framework for discovering panoramas automatically from videos. We have split our design into five specific 
parts, shot detection, homography calculation, quality calculation, panorama discovery, and image stitching. 
The overall flow and how these parts are connected is shown in Figure (\ref{fig:HLD}).